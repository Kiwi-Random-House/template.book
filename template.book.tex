\PassOptionsToPackage{unicode=true}{hyperref} % options for packages loaded elsewhere
\PassOptionsToPackage{hyphens}{url}
%
\documentclass[]{book}
\usepackage{lmodern}
\usepackage{amssymb,amsmath}
\usepackage{ifxetex,ifluatex}
\usepackage{fixltx2e} % provides \textsubscript
\ifnum 0\ifxetex 1\fi\ifluatex 1\fi=0 % if pdftex
  \usepackage[T1]{fontenc}
  \usepackage[utf8]{inputenc}
  \usepackage{textcomp} % provides euro and other symbols
\else % if luatex or xelatex
  \usepackage{unicode-math}
  \defaultfontfeatures{Ligatures=TeX,Scale=MatchLowercase}
\fi
% use upquote if available, for straight quotes in verbatim environments
\IfFileExists{upquote.sty}{\usepackage{upquote}}{}
% use microtype if available
\IfFileExists{microtype.sty}{%
\usepackage[]{microtype}
\UseMicrotypeSet[protrusion]{basicmath} % disable protrusion for tt fonts
}{}
\IfFileExists{parskip.sty}{%
\usepackage{parskip}
}{% else
\setlength{\parindent}{0pt}
\setlength{\parskip}{6pt plus 2pt minus 1pt}
}
\usepackage{hyperref}
\hypersetup{
            pdftitle={Tidy Book Template},
            pdfauthor={Harel Lustiger},
            pdfborder={0 0 0},
            breaklinks=true}
\urlstyle{same}  % don't use monospace font for urls
\usepackage{color}
\usepackage{fancyvrb}
\newcommand{\VerbBar}{|}
\newcommand{\VERB}{\Verb[commandchars=\\\{\}]}
\DefineVerbatimEnvironment{Highlighting}{Verbatim}{commandchars=\\\{\}}
% Add ',fontsize=\small' for more characters per line
\usepackage{framed}
\definecolor{shadecolor}{RGB}{248,248,248}
\newenvironment{Shaded}{\begin{snugshade}}{\end{snugshade}}
\newcommand{\AlertTok}[1]{\textcolor[rgb]{0.94,0.16,0.16}{#1}}
\newcommand{\AnnotationTok}[1]{\textcolor[rgb]{0.56,0.35,0.01}{\textbf{\textit{#1}}}}
\newcommand{\AttributeTok}[1]{\textcolor[rgb]{0.77,0.63,0.00}{#1}}
\newcommand{\BaseNTok}[1]{\textcolor[rgb]{0.00,0.00,0.81}{#1}}
\newcommand{\BuiltInTok}[1]{#1}
\newcommand{\CharTok}[1]{\textcolor[rgb]{0.31,0.60,0.02}{#1}}
\newcommand{\CommentTok}[1]{\textcolor[rgb]{0.56,0.35,0.01}{\textit{#1}}}
\newcommand{\CommentVarTok}[1]{\textcolor[rgb]{0.56,0.35,0.01}{\textbf{\textit{#1}}}}
\newcommand{\ConstantTok}[1]{\textcolor[rgb]{0.00,0.00,0.00}{#1}}
\newcommand{\ControlFlowTok}[1]{\textcolor[rgb]{0.13,0.29,0.53}{\textbf{#1}}}
\newcommand{\DataTypeTok}[1]{\textcolor[rgb]{0.13,0.29,0.53}{#1}}
\newcommand{\DecValTok}[1]{\textcolor[rgb]{0.00,0.00,0.81}{#1}}
\newcommand{\DocumentationTok}[1]{\textcolor[rgb]{0.56,0.35,0.01}{\textbf{\textit{#1}}}}
\newcommand{\ErrorTok}[1]{\textcolor[rgb]{0.64,0.00,0.00}{\textbf{#1}}}
\newcommand{\ExtensionTok}[1]{#1}
\newcommand{\FloatTok}[1]{\textcolor[rgb]{0.00,0.00,0.81}{#1}}
\newcommand{\FunctionTok}[1]{\textcolor[rgb]{0.00,0.00,0.00}{#1}}
\newcommand{\ImportTok}[1]{#1}
\newcommand{\InformationTok}[1]{\textcolor[rgb]{0.56,0.35,0.01}{\textbf{\textit{#1}}}}
\newcommand{\KeywordTok}[1]{\textcolor[rgb]{0.13,0.29,0.53}{\textbf{#1}}}
\newcommand{\NormalTok}[1]{#1}
\newcommand{\OperatorTok}[1]{\textcolor[rgb]{0.81,0.36,0.00}{\textbf{#1}}}
\newcommand{\OtherTok}[1]{\textcolor[rgb]{0.56,0.35,0.01}{#1}}
\newcommand{\PreprocessorTok}[1]{\textcolor[rgb]{0.56,0.35,0.01}{\textit{#1}}}
\newcommand{\RegionMarkerTok}[1]{#1}
\newcommand{\SpecialCharTok}[1]{\textcolor[rgb]{0.00,0.00,0.00}{#1}}
\newcommand{\SpecialStringTok}[1]{\textcolor[rgb]{0.31,0.60,0.02}{#1}}
\newcommand{\StringTok}[1]{\textcolor[rgb]{0.31,0.60,0.02}{#1}}
\newcommand{\VariableTok}[1]{\textcolor[rgb]{0.00,0.00,0.00}{#1}}
\newcommand{\VerbatimStringTok}[1]{\textcolor[rgb]{0.31,0.60,0.02}{#1}}
\newcommand{\WarningTok}[1]{\textcolor[rgb]{0.56,0.35,0.01}{\textbf{\textit{#1}}}}
\usepackage{longtable,booktabs}
% Fix footnotes in tables (requires footnote package)
\IfFileExists{footnote.sty}{\usepackage{footnote}\makesavenoteenv{longtable}}{}
\usepackage{graphicx,grffile}
\makeatletter
\def\maxwidth{\ifdim\Gin@nat@width>\linewidth\linewidth\else\Gin@nat@width\fi}
\def\maxheight{\ifdim\Gin@nat@height>\textheight\textheight\else\Gin@nat@height\fi}
\makeatother
% Scale images if necessary, so that they will not overflow the page
% margins by default, and it is still possible to overwrite the defaults
% using explicit options in \includegraphics[width, height, ...]{}
\setkeys{Gin}{width=\maxwidth,height=\maxheight,keepaspectratio}
\setlength{\emergencystretch}{3em}  % prevent overfull lines
\providecommand{\tightlist}{%
  \setlength{\itemsep}{0pt}\setlength{\parskip}{0pt}}
\setcounter{secnumdepth}{5}
% Redefines (sub)paragraphs to behave more like sections
\ifx\paragraph\undefined\else
\let\oldparagraph\paragraph
\renewcommand{\paragraph}[1]{\oldparagraph{#1}\mbox{}}
\fi
\ifx\subparagraph\undefined\else
\let\oldsubparagraph\subparagraph
\renewcommand{\subparagraph}[1]{\oldsubparagraph{#1}\mbox{}}
\fi

% set default figure placement to htbp
\makeatletter
\def\fps@figure{htbp}
\makeatother

\usepackage{booktabs}
\usepackage{graphicx}
\usepackage{etoolbox}
\makeatletter
\providecommand{\subtitle}[1]{% add subtitle to \maketitle
  \apptocmd{\@title}{\par {\large #1 \par}}{}{}
}
\makeatother
\usepackage[]{natbib}
\bibliographystyle{apalike}

\title{Tidy Book Template}
\providecommand{\subtitle}[1]{}
\subtitle{A Guide for How to Deploy a Book.}
\author{Harel Lustiger}
\date{2020-04-19}

\begin{document}
\maketitle

{
\setcounter{tocdepth}{1}
\tableofcontents
}
\hypertarget{demo}{%
\chapter{Using the Template}\label{demo}}

\hypertarget{create-a-new-book-on-github-using-the-template}{%
\section{Create a New Book on GitHub using the Template}\label{create-a-new-book-on-github-using-the-template}}

\begin{center}\includegraphics[width=0.7\linewidth]{images/a9676ba37e5e4ca8b9489239db8cb8d8} \end{center}

\hypertarget{link-the-book-with-travis}{%
\section{Link the Book with Travis}\label{link-the-book-with-travis}}

\begin{enumerate}
\def\labelenumi{\arabic{enumi}.}
\tightlist
\item
  Go to \href{https://travis-ci.org/}{travis-ci.org} (not travis-ci.com);
\item
  Authorise Travis access to the book's GitHub repo; and
\item
  Toggle legacy service integration for the book's GitHub repo.
\end{enumerate}

\begin{center}\includegraphics[width=0.7\linewidth]{images/baef0bd157faee7c5dac5dbe2712c3b6} \end{center}

\hypertarget{add-github-api-to-travis}{%
\section{Add GitHub API to Travis}\label{add-github-api-to-travis}}

\begin{enumerate}
\def\labelenumi{\arabic{enumi}.}
\tightlist
\item
  Generate GitHub Personal Access Token (PAT) by either:
\end{enumerate}

\begin{itemize}
\tightlist
\item
  Following the instructions provided on GitHub Help pages; or
\item
  Running the command \texttt{usethis::browse\_github\_token()}.
\end{itemize}

\begin{enumerate}
\def\labelenumi{\arabic{enumi}.}
\setcounter{enumi}{1}
\tightlist
\item
  Add PAT as an environment variable named \texttt{GITHUB\_PAT} within project setting.
\end{enumerate}

\begin{center}\includegraphics[width=0.7\linewidth]{images/0fcd976cfcba59500407c33c05ecd4b0} \end{center}

\hypertarget{set-ssh-key-pair-via-travis}{%
\section{Set SSH key pair via travis}\label{set-ssh-key-pair-via-travis}}

Only needed when deploying from builds on Travis CI or GitHub Actions.

\begin{Shaded}
\begin{Highlighting}[]
\CommentTok{# Install travis R package}
\NormalTok{remotes}\OperatorTok{::}\KeywordTok{install_github}\NormalTok{(}\StringTok{"ropenscilabs/travis"}\NormalTok{)}

\CommentTok{# Generate SSH}
\NormalTok{travis}\OperatorTok{::}\KeywordTok{browse_travis_token}\NormalTok{()}
\NormalTok{travis}\OperatorTok{::}\KeywordTok{use_travis_deploy}\NormalTok{()}
\end{Highlighting}
\end{Shaded}

\hypertarget{trigger-travis-to-deploy-the-book}{%
\section{Trigger Travis to Deploy the Book}\label{trigger-travis-to-deploy-the-book}}

Trigger the first deployment on the \emph{master} and \emph{develop} branches.
You can do it either:

\begin{itemize}
\tightlist
\item
  Directly from GitHub by pushing changes into a new branch call \emph{develop}; or
\item
  Through \href{https://www.sourcetreeapp.com/}{SourceTree} by:

  \begin{enumerate}
  \def\labelenumi{\arabic{enumi}.}
  \tightlist
  \item
    Cloning the repo to local computer through SourceTree;
  \item
    Initiating Git-flow; and
  \item
    Starting a new release named \emph{book-inception}.
  \end{enumerate}
\end{itemize}

At this stage of using the template, there are several items we can update:

\begin{enumerate}
\def\labelenumi{\arabic{enumi}.}
\tightlist
\item
  Rename \texttt{template.book.Rproj} to \texttt{\textless{}book-name\textgreater{}.Rproj}; and
\item
  Update the \textbf{Title}, \textbf{Description} and \textbf{Date} fields in DESCRIPTION.
\end{enumerate}

Finally commit the changes:

\begin{itemize}
\tightlist
\item
  If you use GitHub website, then push the changes to the \emph{develop} branch and
  merge the \emph{master} branch.
\item
  If you use SourceTree, then finish the release and push changes to remote.
\end{itemize}

\begin{center}\includegraphics[width=0.7\linewidth]{images/317f5871bf095360684ad70f23eec266} \end{center}

The first deployment takes \textasciitilde{}9 minutes to complete. At the end of a
successful run, two new branches appear in the GitHub repo: \emph{gh-pages} and
\emph{gh-preview}.

\hypertarget{link-the-book-with-netlify}{%
\section{Link the Book with Netlify}\label{link-the-book-with-netlify}}

\begin{enumerate}
\def\labelenumi{\arabic{enumi}.}
\tightlist
\item
  Go to \url{https://app.netlify.com/}; and
\item
  Follow the illustrations.
\end{enumerate}

\begin{center}\includegraphics[width=0.7\linewidth]{images/db5b10aba61256bb01d9731ac2cd0830} \includegraphics[width=0.7\linewidth]{images/75b90a0fdcfbad8c4cea92c8c0ba013a} \includegraphics[width=0.7\linewidth]{images/3be8c926bcd3b95f738a64ca8193c3e8} \end{center}

Modify the following by clicking on ``Build settings'' at the right navigation
bar. Then, click ``Edit settings'' under ``Deploy contexts'':

\begin{enumerate}
\def\labelenumi{\arabic{enumi}.}
\tightlist
\item
  Set ``Production branch'' to gh-pages;
\item
  Set ``Branch deploys'' to ``Let me add individual branches''; and
\item
  Add gh-preview under ``Additional branches''.
\end{enumerate}

\begin{center}\includegraphics[width=0.7\linewidth]{images/da630a35606ef9313dae374c8e8f1cd7} \includegraphics[width=0.7\linewidth]{images/5aae39e4fb32a148c6d8a59f2c87b5de} \end{center}

\hypertarget{update-site-name-on-netlify}{%
\section{Update Site Name on Netlify}\label{update-site-name-on-netlify}}

\begin{center}\includegraphics[width=0.7\linewidth]{images/4969813bdca8f1b44bd7b7cd5266ff36} \end{center}

\hypertarget{update-netlify-fields-within-description}{%
\section{Update Netlify Fields within DESCRIPTION}\label{update-netlify-fields-within-description}}

\begin{enumerate}
\def\labelenumi{\arabic{enumi}.}
\tightlist
\item
  Update \textbf{NetlifyURL} with the site URL; and
\item
  Update \textbf{NetlifyID} with API ID.
\end{enumerate}

\begin{center}\includegraphics[width=0.7\linewidth]{images/0d6fe5708049ecd979b23cc75c2ff9dc} \includegraphics[width=0.7\linewidth]{images/63fd32aaaf30f529e9f1e458e18fbd87} \end{center}

\hypertarget{update-github-readme-file}{%
\section{Update GitHub README File}\label{update-github-readme-file}}

\begin{enumerate}
\def\labelenumi{\arabic{enumi}.}
\tightlist
\item
  Render README.Rmd in R
\item
  Push changes
\end{enumerate}

\begin{center}\includegraphics[width=0.7\linewidth]{images/1d6ec6730103d6654ec4dc8e03c642be} \includegraphics[width=0.7\linewidth]{images/1db9c26c547348c10d24a7f86ce3d3ca} \end{center}

Congratulations, you've made it!

\hypertarget{part-front-matter}{%
\part*{Front Matter}\label{part-front-matter}}
\addcontentsline{toc}{part}{Front Matter}

\hypertarget{title-page}{%
\chapter*{Title Page}\label{title-page}}
\addcontentsline{toc}{chapter}{Title Page}

\hypertarget{dedication}{%
\chapter*{Dedication}\label{dedication}}
\addcontentsline{toc}{chapter}{Dedication}

\hypertarget{foreword}{%
\chapter*{Foreword}\label{foreword}}
\addcontentsline{toc}{chapter}{Foreword}

\hypertarget{preface}{%
\chapter*{Preface}\label{preface}}
\addcontentsline{toc}{chapter}{Preface}

\hypertarget{colophon}{%
\section*{Colophon}\label{colophon}}
\addcontentsline{toc}{section}{Colophon}

This book was written in \href{http://www.rstudio.com/ide/}{RStudio} using
\href{http://bookdown.org/}{bookdown}. The \href{http://mastering-shiny.org/}{website} is
hosted with \href{http://netlify.com/}{netlify}, and automatically updated after
every commit by \href{https://travis-ci.org/}{travis-ci}. The complete source is
available from \href{https://github.com/Kiwi-Random-House/template.book}{GitHub}.

This version of the book was built with R version 3.6.2 (2017-01-27) and the following
packages:

\begin{longtable}[]{@{}lll@{}}
\toprule
package & version & source\tabularnewline
\midrule
\endhead
ggplot2 & 3.2.1 & CRAN (R 3.6.2)\tabularnewline
\bottomrule
\end{longtable}

\hypertarget{table-of-contents}{%
\chapter*{Table of Contents}\label{table-of-contents}}
\addcontentsline{toc}{chapter}{Table of Contents}

\hypertarget{list-of-abbreviations}{%
\chapter*{List of Abbreviations}\label{list-of-abbreviations}}
\addcontentsline{toc}{chapter}{List of Abbreviations}

\hypertarget{part-text-body}{%
\part*{Text Body}\label{part-text-body}}
\addcontentsline{toc}{part}{Text Body}

\hypertarget{intro}{%
\chapter{Introduction}\label{intro}}

\hypertarget{terminology}{%
\section{Terminology}\label{terminology}}

The following terms come up repeatedly in discussion of machine learning\citep{GoogleRulesML}:

\begin{itemize}
\tightlist
\item
  \textbf{Instance}: The thing about which you want to make a prediction. For
  example, the instance might be a web page that you want to classify as either
  ``about cats'' or ``not about cats''.
\item
  \textbf{Label}: An answer for a prediction task; either the answer produced by a
  machine learning system, or the right answer supplied in training data. For
  example, the label for a web page might be ``about cats''.
\item
  \textbf{Feature}: A property of an instance used in a prediction task. For example,
  a web page might have a feature ``contains the word `cat'\,''.
\item
  \textbf{Feature Column}: A set of related features, such as the set of all possible
  countries in which users might live. An example may have one or more features
  present in a feature column. ``Feature column'' is Google-specific terminology. A
  feature column is referred to as a ``namespace'' in the VW system (at
  Yahoo/Microsoft), or a field. Example: An instance (with its features) and a
  label.
\item
  \textbf{Model}: A statistical representation of a prediction task. You train a
  model on examples then use the model to make predictions.
\item
  \textbf{Metric}: A number that you care about. May or may not be directly
  optimized.
\item
  \textbf{Objective}: A metric that your algorithm is trying to optimize.
\item
  \textbf{Pipeline}: The infrastructure surrounding a machine learning algorithm.
  Includes gathering the data from the front end, putting it into training data
  files, training one or more models, and exporting the models to production.
\end{itemize}

\hypertarget{useful-materials}{%
\section{Useful Materials}\label{useful-materials}}

\begin{itemize}
\tightlist
\item
  \href{https://bookdown.org/yihui/bookdown/}{\texttt{bookdown}: Authoring Books and Technical Documents with R Markdown}
\item
  \href{https://crsh.github.io/papaja_man/}{\texttt{papaja}: Reproducible APA manuscripts with R Markdown}
\end{itemize}

\hypertarget{prerequisites}{%
\section{Prerequisites}\label{prerequisites}}

\hypertarget{example}{%
\section{Example}\label{example}}

You can label chapter and section titles using \texttt{\{\#label\}} after them, e.g., we can reference Chapter \ref{intro}. If you do not manually label them, there will be automatic labels anyway, e.g., Chapter \ref{methods}.

Figures and tables with captions will be placed in \texttt{figure} and \texttt{table} environments, respectively.

\begin{figure}

{\centering \includegraphics[width=0.8\linewidth]{201-intro_files/figure-latex/nice-fig-1} 

}

\caption{Here is a nice figure!}\label{fig:nice-fig}
\end{figure}

Reference a figure by its code chunk label with the \texttt{fig:} prefix, e.g., see Figure \ref{fig:nice-fig}. Similarly, you can reference tables generated from \texttt{knitr::kable()}, e.g., see Table \ref{tab:nice-tab}.

\begin{table}

\caption{\label{tab:nice-tab}Here is a nice table!}
\centering
\begin{tabular}[t]{rrrrl}
\toprule
Sepal.Length & Sepal.Width & Petal.Length & Petal.Width & Species\\
\midrule
5.1 & 3.5 & 1.4 & 0.2 & setosa\\
4.9 & 3.0 & 1.4 & 0.2 & setosa\\
4.7 & 3.2 & 1.3 & 0.2 & setosa\\
4.6 & 3.1 & 1.5 & 0.2 & setosa\\
5.0 & 3.6 & 1.4 & 0.2 & setosa\\
\addlinespace
5.4 & 3.9 & 1.7 & 0.4 & setosa\\
4.6 & 3.4 & 1.4 & 0.3 & setosa\\
5.0 & 3.4 & 1.5 & 0.2 & setosa\\
4.4 & 2.9 & 1.4 & 0.2 & setosa\\
4.9 & 3.1 & 1.5 & 0.1 & setosa\\
\addlinespace
5.4 & 3.7 & 1.5 & 0.2 & setosa\\
4.8 & 3.4 & 1.6 & 0.2 & setosa\\
4.8 & 3.0 & 1.4 & 0.1 & setosa\\
4.3 & 3.0 & 1.1 & 0.1 & setosa\\
5.8 & 4.0 & 1.2 & 0.2 & setosa\\
\addlinespace
5.7 & 4.4 & 1.5 & 0.4 & setosa\\
5.4 & 3.9 & 1.3 & 0.4 & setosa\\
5.1 & 3.5 & 1.4 & 0.3 & setosa\\
5.7 & 3.8 & 1.7 & 0.3 & setosa\\
5.1 & 3.8 & 1.5 & 0.3 & setosa\\
\bottomrule
\end{tabular}
\end{table}

\hypertarget{literature}{%
\chapter{Literature}\label{literature}}

Here is a review of existing methods.

\hypertarget{methods}{%
\chapter{Methods}\label{methods}}

We describe our methods in this chapter.

\hypertarget{applications}{%
\chapter{Applications}\label{applications}}

Some \emph{significant} applications are demonstrated in this chapter.

\hypertarget{example-one}{%
\section{Example one}\label{example-one}}

\hypertarget{example-two}{%
\section{Example two}\label{example-two}}

\hypertarget{final-words}{%
\chapter{Final Words}\label{final-words}}

We have finished a nice book.

\hypertarget{part-back-matter}{%
\part*{Back Matter}\label{part-back-matter}}
\addcontentsline{toc}{part}{Back Matter}

\hypertarget{appendix}{%
\chapter*{Appendix}\label{appendix}}
\addcontentsline{toc}{chapter}{Appendix}

\bibliography{references.bib}

\end{document}
